% !TEX program = xelatex

\documentclass[10pt, dvipsnames]{beamer}

\usetheme[progressbar=frametitle]{metropolis}
\usepackage{appendixnumberbeamer}
\usepackage{booktabs}
\usepackage[ngerman]{babel}
\usepackage[ngerman]{varioref}
\usepackage[scale=2]{ccicons}
\usepackage[absolute,overlay]{textpos}
\usepackage{graphicx}
\usepackage{amsmath}
\usepackage{natbib}
\usepackage{appendixnumberbeamer}
\usepackage{listings} %% for R Code

%%% BibTex Style
\setcitestyle{authoryear, open = { ( }, close = { ) }}
\def\bibfont{\small} % smaller bibliography

%%% Listings options
\lstdefinelanguage{Renhanced}[]{R}{
otherkeywords={!,!=,~,*,\&,\%/\%,\%*\%,\%\%,<-,<<-, ::},
morekeywords={},
deletekeywords={hist, runif, plot, data, family , read.table, read, check, text, file, attributes, quote, missing, c, list, any, which, na, deparse, structure, install},
alsoletter={.\%},%
alsoother={:_\$}}

\lstset{
language=Renhanced,                     % the language of the code
basicstyle=\small\ttfamily, % the size of the fonts that are used for the code
numbers=left,                   % where to put the line-numbers
numberstyle=\tiny\color{Blue},  % the style that is used for the line-numbers
stepnumber=1,                   % the step between two line-numbers. If it is 1, each line will be numbered
numbersep=10pt,                  % how far the line-numbers are from the code
backgroundcolor=\color{white},  % choose the background color. You must add \usepackage{color}
showspaces=false,               % show spaces adding particular underscores
showstringspaces=false,         % underline spaces within strings
showtabs=false,                 % show tabs within strings adding particular underscores
frame=false,                   % adds a frame around the code
rulecolor=\color{black},        % if not set, the frame-color may be changed on line-breaks within not-black text (e.g. commens (green here))
tabsize=2,                      % sets default tabsize to 2 spaces
captionpos=b,                   % sets the caption-position to bottom
breaklines=true,                % sets automatic line breaking
breakatwhitespace=false,        % sets if automatic breaks should only happen at whitespace
keywordstyle=\color{RoyalBlue},      % keyword style
commentstyle=\color{YellowGreen},   % comment style
stringstyle=\color{ForestGreen}      % string literal style
}

\renewcommand{\lstlistingname}{Code-Chunk}

%%% New commands
\newcommand{\li}{\lstinline}

\addto{\captionsngerman}{%
\renewcommand*{\contentsname}{Inhalt}
\renewcommand*{\listfigurename}{Figures}
\renewcommand*{\listtablename}{Table}
\renewcommand*{\figurename}{Figure}
}

\setsansfont[BoldFont={Fira Sans SemiBold}]{Fira Sans Book}

\usepackage{pgfplots}
\usepgfplotslibrary{dateplot}

\usepackage{xspace}
\newcommand{\themename}{\textbf{\textsc{metropolis}}\xspace}
\title{bamlss.vis}
\subtitle{An R Package to Interactively Analyze and Visualize Bayesian Additive Models for Location, Scale and Shape (bamlss) Using the Shiny Framework}
\date{7. December 2017}
\author{Stanislaus Stadlmann}
\institute{Georg-August University of Göttingen}
\titlegraphic{
\vspace{-0.2cm}~%
\hspace*{1.25cm}~%
\includegraphics[height = 1.4cm]{images/00_logo.png}
}

\begin{document}

\maketitle

\begin{frame}{Table of contents}
  \setbeamertemplate{section in toc}[sections numbered]
  \tableofcontents
\end{frame}

\section{Introduction}
\begin{frame}{Introduction}
  \textbf{Distributional Regression} \\
  \begin{itemize}
    \item An emerging field in regression methods
    \item Each parameter of a response distribution beyond the mean can be modeled using a set of predictors
    \pause
    \item Notable frameworks:
    \begin{enumerate}
      \item Generalized Additive Models for Location, Scale and Shape, coined by \citet*{gamlss2001}
      \item Bayesian Additive Models for Location, Scale and Shape, coined by \citet*{bamlss2017}
    \end{enumerate}
    \item Differences: Estimation techniques
  \end{itemize}
\end{frame}

\begin{frame}{Introduction}
  \begin{columns}[T,onlytextwidth]
    \column{0.35\textwidth}
    \metroset{block=fill}
    \begin{exampleblock}{Response}
      Let $y \sim D(\theta_1, \ldots, \theta_K)$
    \end{exampleblock}
  \end{columns}
  \textbf{Problem} \\
  \begin{itemize}
    \item Often, distribution parameters $h(\eta_l) = \theta_l$ do not directly equate to $E(y)$, $Var(y)$
    \pause
    \item Therefore hard to interpret effects on distribution moments because:
    \begin{enumerate}
      \item Link function $h_l(\cdot)$ transforms effects
      \item Transformed effects are for parameters $\theta_l$ which often do not directly equate moments
    \end{enumerate}
  \end{itemize}
\end{frame}

\begin{frame}[t]
  \textbf{Problem} \\
  \begin{itemize}
    \item Thus: Package needed which
    \begin{enumerate}
      \item Makes it easy to graphically display and compare predicted distributions
      \item Displays the influence of a covariate on the distributional moments
    \end{enumerate}
    \item $\Rightarrow$ \li{bamlss.vis} was born, solving these problems in R with a Shiny App.
  \end{itemize}
\end{frame}

\section{Motivating BAMLSS}

\begin{frame}{Additive Models (AM)}
  \textbf{Overview} \\
  \begin{itemize}
    \item Proposed by \citet*{friedman1981projection}
    \item Dependent variable $y$ is related to non-parametric effects in an additive way
  \end{itemize}
  \pause
  \textbf{Model specification}
  \begin{equation*}
    \begin{split}
      y_i & = f_1(z_{i1}) + f_2(z_{i2}) + \ldots + f_k(z_{ik}) + \epsilon_i \quad \text{(only nonparametric effects)} \\
      y_i & = \sum\limits_{j = 1}^K f_j(z_{ij}) + \sum\limits_{l = 1}^Q \beta_l x_{il} + \epsilon_i \quad \text{(with parametric effects)}
    \end{split}
  \end{equation*}
  \pause
  \textbf{Why additive?}
  \begin{itemize}
    \item Curse of dimensionality
    \item Easier to separate covariate effects
  \end{itemize}
\end{frame}

\begin{frame}{Structured Additive Regression (STAR) Models}
  \textbf{Motivation} \\
  \begin{itemize}
    \item AM allow for non-parametric effects, but sometimes even more flexibility is needed
    \item STAR \citep{fahrmeir2003} also support structured terms, which include:
    \pause
    \begin{enumerate}
      \item Nonlinear effects of a single variable
      \item Spatial effects of location index $s$
      \item Interactions between a continuous covariate and a categorical variable
      \item Nonlinear interactions between two continuous covariates
      \item Random Effects with intercept $\nu_0$ and slope $\nu_j$ deviations from main effects
    \end{enumerate}
  \end{itemize}
  \pause
  \textbf{Model specification}
  \begin{equation*}
    y_i = \underbrace{\kappa_i^{add}}_{\text{AM components}} + f_{struc}(\mathbf{v}_{i1}) + \epsilon_i
  \end{equation*}
  where $\mathbf{v}_{1}$ can be a one- or multidimensional variable.
\end{frame}

\begin{frame}{Generalized STAR Models}
  \textbf{Motivation} \\
  \begin{itemize}
    \item AM and STAR assume normalty and directly model $E(y)$
    \item Generalized STAR models use link function $g(\cdot)$ of Generalized Linear Models
    \item Adds ability to model $E(y)$ of all exponential families, e.g. binomial or poisson distribution
  \end{itemize}
  \textbf{Model specification}
  \begin{equation*}
    \begin{split}
      g(\mu_i) & = \eta_i \\
      \eta_i & = f_1(z_{i1}) + \ldots + f_K(z_{iK}) + \beta_0 + \beta_1 x_{i1} + \ldots + \beta_Q x_{iQ}
    \end{split}
  \end{equation*}
\end{frame}

\begin{frame}{Structured Additive Distributional Regression}
  \textbf{Motivation} \\
  \begin{itemize}
    \item Often, more than just the location of a distribution is of interest
    \item Scale/Shape (Variance, Kurtosis) might also be dependent on covariates
    \item Structured Additive Distributional Regression allows modeling of all distributional parameters $\theta_l$
  \end{itemize}
  \textbf{Model specification}\\
  Let $y \sim D(\theta_1, \ldots, \theta_K)$. Then:
\begin{equation*}
  \begin{split}
  g_l(\boldsymbol{\theta}_l) & = \boldsymbol{\eta}_l \\
  & = f_{1l}(\mathbf{X} ; \boldsymbol{\beta}_{1l}) + \ldots + f_{Q_{l}l}(\mathbf{X}) ; \boldsymbol{\beta}_{Q_{l}l})
\end{split}
\end{equation*}
where every $\theta_l$ can be modeled with effect types of different subsets of $\mathbf{X}$.
\end{frame}

\begin{frame}{Bayesian Models for Location, Scale and Shape}
  overview...
\end{frame}

\begin{frame}{Thanks!}
  Thanks for your attention!
\end{frame}

\appendix

\begin{frame}[allowframebreaks]{References}
  \bibliography{bibliography}
  \bibliographystyle{plainnat}
\end{frame}

\end{document}
