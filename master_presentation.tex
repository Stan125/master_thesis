% !TEX program = xelatex

\documentclass[10pt, dvipsnames]{beamer}

\usetheme[progressbar=frametitle]{metropolis}
\usepackage{appendixnumberbeamer}
\usepackage{booktabs}
\usepackage[ngerman]{babel}
\usepackage[ngerman]{varioref}
\usepackage[scale=2]{ccicons}
\usepackage[absolute,overlay]{textpos}
\usepackage{graphicx}
\usepackage{amsmath}
\usepackage[justification=centering]{caption}
\usepackage[autostyle]{csquotes}
\usepackage{natbib}
\usepackage{appendixnumberbeamer}
\usepackage{listings} %% for R Code

\setbeamercolor{background canvas}{bg=white}

%%% BibTex Style
\setcitestyle{authoryear, open = { ( }, close = { ) }}
\def\bibfont{\small} % smaller bibliography

%%% Listings options
\lstdefinelanguage{Renhanced}[]{R}{
otherkeywords={!,!=,~,*,\&,\%/\%,\%*\%,\%\%,<-,<<-, ::},
morekeywords={},
deletekeywords={hist, runif, plot, data, family , read.table, read, check, text, file, attributes, quote, missing, c, list, any, which, na, deparse, structure, install},
alsoletter={.\%},%
alsoother={:_\$}}

\lstset{
language=Renhanced,                     % the language of the code
basicstyle=\small\ttfamily, % the size of the fonts that are used for the code
numbers=left,                   % where to put the line-numbers
numberstyle=\tiny\color{Blue},  % the style that is used for the line-numbers
stepnumber=1,                   % the step between two line-numbers. If it is 1, each line will be numbered
numbersep=10pt,                  % how far the line-numbers are from the code
backgroundcolor=\color{white},  % choose the background color. You must add \usepackage{color}
showspaces=false,               % show spaces adding particular underscores
showstringspaces=false,         % underline spaces within strings
showtabs=false,                 % show tabs within strings adding particular underscores
frame=false,                   % adds a frame around the code
rulecolor=\color{black},        % if not set, the frame-color may be changed on line-breaks within not-black text (e.g. commens (green here))
tabsize=2,                      % sets default tabsize to 2 spaces
captionpos=b,                   % sets the caption-position to bottom
breaklines=true,                % sets automatic line breaking
breakatwhitespace=false,        % sets if automatic breaks should only happen at whitespace
keywordstyle=\color{RoyalBlue},      % keyword style
commentstyle=\color{YellowGreen},   % comment style
stringstyle=\color{ForestGreen}      % string literal style
}

\renewcommand{\lstlistingname}{Code-Chunk}

%%% New commands
\newcommand{\li}{\lstinline}

\addto{\captionsngerman}{%
\renewcommand*{\contentsname}{Inhalt}
\renewcommand*{\listfigurename}{Figures}
\renewcommand*{\listtablename}{Table}
\renewcommand*{\figurename}{Figure}
}

\setsansfont[BoldFont={Fira Sans SemiBold}]{Fira Sans Book}

\usepackage{pgfplots}
\usepgfplotslibrary{dateplot}

\usepackage{xspace}
\newcommand{\themename}{\textbf{\textsc{metropolis}}\xspace}
\title{bamlss.vis}
\subtitle{An R Package to Interactively Analyze and Visualize Bayesian Additive Models for Location, Scale and Shape (bamlss) Using the Shiny Framework}
\date{7. December 2017}
\author{Stanislaus Stadlmann}
\institute{Georg-August University of Göttingen}
\titlegraphic{
\vspace{-0.2cm}~%
\hspace*{1.25cm}~%
\includegraphics[height = 1.4cm]{images/00_logo.png}
}

\begin{document}

\maketitle

\begin{frame}{Table of contents}
  \setbeamertemplate{section in toc}[sections numbered]
  \tableofcontents
\end{frame}

\section{Introduction}
\begin{frame}{Introduction}
  \textbf{Distributional Regression} \\
  \begin{itemize}
    \item An emerging field in regression methods
    \item Each parameter of a response distribution beyond the mean can be modeled using a set of predictors
    \pause
    \item Notable frameworks:
    \begin{enumerate}
      \item Generalized Additive Models for Location, Scale and Shape, coined by \citet*{gamlss2001}
      \item Bayesian Additive Models for Location, Scale and Shape, coined by \citet*{bamlss2017}
    \end{enumerate}
    \item Differences: Estimation techniques - Likelihood/Bayes
  \end{itemize}
\end{frame}

\begin{frame}{Introduction}
  \textbf{bamlss.vis} \\
  \begin{itemize}
    \item R package based on the Shiny framework
    \item Built upon R package \li{bamlss}
    \item Requires a fitted \li{bamlss} object
    \item Yields the abilities to
    \begin{enumerate}
      \item visualize predictions for user-chosen covariate combinations
      \item visualize the influence of a certain covariate on distributional moments
    \end{enumerate}
  \end{itemize}
\end{frame}

\section{BAMLSS ancestry}

\begin{frame}{Additive Models (AM)}
  \textbf{Overview} \\
  \begin{itemize}
    \item Proposed by \citet*{friedman1981projection}
    \item Dependent variable $y$ is related to non-parametric effects in an additive way
  \end{itemize}
  \pause
  \textbf{Model specification}
  \begin{equation*}
    \begin{split}
      y_i & = f_1(z_{i1}) + f_2(z_{i2}) + \ldots + f_K(z_{iK}) + \epsilon_i \quad \text{(only nonparametric effects)} \\
      y_i & = \sum\limits_{j = 1}^K f_j(z_{ij}) + \sum\limits_{l = 1}^Q \beta_l x_{il} + \epsilon_i \quad \text{(with parametric effects)}
    \end{split}
  \end{equation*}
  \pause
  \textbf{Why additive?}
  \begin{itemize}
    \item Curse of dimensionality
    \item Easier to separate covariate effects
  \end{itemize}
\end{frame}

\begin{frame}{Structured Additive Regression (STAR) Models}
  \textbf{Motivation} \\
  \begin{itemize}
    \item AM allow for non-parametric effects, but sometimes even more flexibility is needed
    \item STAR \citep{fahrmeir2003} also support structured terms, which include:
    \pause
    \begin{enumerate}
      \item Nonlinear effects of a single variable
      \item Spatial effects of location index $s$
      \item Interactions between a continuous covariate and a categorical variable
      \item Nonlinear interactions between two continuous covariates
      \item Random Effects with intercept $\nu_0$ and slope $\nu_j$ deviations from main effects
    \end{enumerate}
  \end{itemize}
  \pause
  \textbf{Model specification}
  \begin{equation*}
    y_i = \underbrace{\kappa_i^{add}}_{\text{AM components}} + \: f_{struc}(\mathbf{z}_{iF}) + \epsilon_i
  \end{equation*}
  where $\mathbf{z}_{F}$ can be a one- or multidimensional variable.
\end{frame}

\begin{frame}{Generalized STAR Models}
  \textbf{Motivation} \\
  \begin{itemize}
    \item AM and STAR assume normalty and directly model $E(y)$
    \item Generalized STAR models use link function $g(\cdot)$ of Generalized Linear Models
    \item Adds ability to model $E(y)$ of all exponential families, e.g. binomial or poisson distribution
  \end{itemize}
  \pause
  \textbf{Model specification}
  \begin{equation*}
    \begin{split}
      g(\mu_i) & = \eta_i \\
      \eta_i & = f_1(\mathbf{z}_{i1}) + \ldots + f_J(\mathbf{z}_{iJ})
    \end{split}
  \end{equation*}
  where $f_j(\cdot)$ can be any structured effect.
\end{frame}

\begin{frame}{Structured Additive Distributional Regression}
  \textbf{Motivation} \\
  \begin{itemize}
    \item Often, more than just the location (Expected Value) of a distribution is of interest
    \item Scale/Shape (Variance, Kurtosis) might also be dependent on covariates
    \item Structured Additive Distributional Regression allows modeling of all distributional parameters $\theta_l$
  \end{itemize}
  \pause
  \textbf{Model specification}\\
  Let $y \sim D(\theta_1, \ldots, \theta_L)$. Then:
  \begin{equation*}
    \begin{split}
      g_{l}(\theta_{il}) & = \eta_{il} \\
      \eta_{il} & = f_{1l}(\mathbf{z}_{i1l}) + \ldots + f_{J_{l}l}(\mathbf{z}_{iJ_{l}l})
    \end{split}
  \end{equation*}
  where every $\theta_l$ can be modeled with effect types of different subsets of $\mathbf{Z}$.
\end{frame}

\begin{frame}{Bayesian Models for Location, Scale and Shape}
  \textbf{Overview} \\
  \begin{itemize}
    \item Coined by \citet*{bamlss2017}
    \item Bayesian variant of Structured Additive Distributional Regression
    \item \enquote{Full} Bayesian inference with
    \begin{enumerate}
      \item Posterior distribution maximisation and
      \item Markov Chain Monte Carlo Sampling
    \end{enumerate}
  \end{itemize}
  \pause
  \textbf{Differences/Advantages over GAMLSS} \\
  \begin{itemize}
    \item Valid credible intervals in comparison to CI based on asymptotics
    \item Structured Additive Effects
    \item Support of Multivariate Distributions
  \end{itemize}
  \textbf{but} \\
  \begin{itemize}
    \item Slower estimation
  \end{itemize}
\end{frame}

\section{Motivation for bamlss.vis}

\begin{frame}{Motivation for bamlss.vis}
  \textbf{Problem} \\
  \begin{itemize}
    \item Often, distribution parameters $\theta_l$ do not directly equate to $E(y)$, $Var(y)$
    \pause
    \item Therefore hard to know influence of covariates on moments because:
    \begin{enumerate}
      \item Link function $h_l(\cdot)$ transforms effects
      \item Transformed effects are for parameters $\theta_l$, which are often not directly moments
    \end{enumerate}
  \end{itemize}
\end{frame}

\begin{frame}{Motivation for bamlss.vis}
  \textbf{An example} \\
  Consider the censored normal distribution $y^* \sim CN(\mu = 0, \sigma^2 = 1)$ with cut-off point $a = 0$. \\[0.5cm]
  \begin{columns}[T, onlytextwidth]
    \column{0.5\textwidth}
    \textbf{The Problem} \\
    \begin{itemize}
      \item Blue line depicts the expected value
      \item Parameters $\mu$ and $\sigma^2$ are not the first two moments of the CN distribution!
    \end{itemize}
    $\Rightarrow$ Any predicted parameters $\hat{\mu}$ and $\hat{\sigma^2}$ need transformation to $E(y^*) / V(y^*)$

    \column{0.5\textwidth}
    \begin{figure}
      \begin{centering}
        \includegraphics[scale = 0.65]{images/030_cnorm_plot.pdf}
        \caption{PDF of CN with expected value as blue line.}
        \label{cnorm_plot}
      \end{centering}
    \end{figure}
  \end{columns}
\end{frame}

\begin{frame}[t]{Motivation for bamlss.vis}
  \textbf{Solution} \\
  \begin{itemize}
    \item Thus: Package needed which
    \begin{enumerate}
      \item Makes it easy to graphically display and compare predicted distributions
      \item Displays the influence of a covariate on the distributional moments
    \end{enumerate}
    \item $\Rightarrow$ \li{bamlss.vis} was born, solving these problems in R with a Shiny App.
  \end{itemize}
\end{frame}

\section{Case-Study}

\begin{frame}{Case-Study}
  $\Rightarrow$ Use real data to illustrate \li{bamlss.vis}' capabilities \\[0.5cm]
  \textbf{The Data}
  \begin{itemize}
    \item \li{Wage} dataset, by \citet*{uscensus}
    \item Depicts yearly income in 1000\$ of males from the US East Coast based on:
    \begin{enumerate}
      \item \textbf{age}
      \item \textbf{year}
      \item \textbf{race}
      \item \textbf{education}
      \item \textbf{health}
    \end{enumerate}
  \end{itemize}
\end{frame}

\begin{frame}{Case-Study}
  \textbf{First look}
  \begin{figure}
    \begin{centering}
      \includegraphics[scale = 0.55]{images/030_wage_plot.pdf}
      \caption{Gaussian kernel density estimates for wages split up by education level.}
      \label{wage_density}
    \end{centering}
  \end{figure}
  $\Rightarrow$ Model both $\mu$ and $\sigma^2$ depending on education level
\end{frame}

\begin{frame}[fragile]{Case-Study}
  \textbf{The model}
  \begin{lstlisting}[caption = R code for fitting the bamlss based on Wage dataset, label = bamlss_fitting]
  cnorm_model <- bamlss(
  list(wage ~ s(age) + race + year + education + health,
       sigma ~ s(age) + race + year + education + health),
  data = wage_sub,
  family = cnorm_bamlss()
  )
    \end{lstlisting}
  \end{frame}

\section{bamlss.vis}

  \begin{frame}{bamlss.vis}
    \centering
    Let's start up \li{bamlss.vis}!
  \end{frame}

  \begin{frame}[fragile]{bamlss.vis}
    \textbf{Installation} \\
    You can install \li{bamlss.vis} today! Run the following code:
    \begin{lstlisting}
    if (!require(devtools))
      install.packages("devtools")
    devtools::install_github("Stan125/bamlss.vis")
    \end{lstlisting}

  \end{frame}

  \begin{frame}{Thanks!}
    \centering
    Thanks for your attention!
  \end{frame}

  \appendix

  \begin{frame}[allowframebreaks]{References}
    \bibliography{bibliography}
    \bibliographystyle{plainnat}
  \end{frame}

  \end{document}
